\documentclass{article}
\usepackage[utf8]{inputenc}
\usepackage{csquotes}
\usepackage{todonotes}

\title{Quick and Efficient Cross-chain Asset Exchange using Multi-signature Transactions and Witnesses}
\author{Martijn de Vos}
\date{April 2021}

\begin{document}

\maketitle

\section{Introduction}
TODO introduce topic

A common approach to exchange assets residing on different blockchains is by using \emph{atomic swaps}~\cite{herlihy2018atomic}.
An atomic swap is a coordination task where assets are either exchanged between parties or nothing happens.
This makes it a relatively secure mechanism to trade values across isolated blockchain environments.
The \enquote{traditional} atomic swap between two parties, Alice and Bob, uses hash-lock and time-lock transaction primitives to refund assets to one of the trading parties when the counterparty becomes inactive.

While atomic swaps seem to be the standard approach for cross-chain asset exchange without trusted intermediary (and also are foundational for many layer-two solutions~\cite{gudgeon2019sok}), this approach has deficiencies for the involved parties.
First, not all blockchain have native support for atomic swaps, and some protocols do not support hash- or time-locks.
Second, recent work shows that atomic swaps are susceptible for collusion with the operators of the involved blockchains~\cite{tsabary2020mad}.
In particular, a malicious user can bribe a miner to ignore the claim transaction of the counterparty, effectively confiscating their funds.
Third, atomic swaps are proven to be unfair for one of the parties since it effectively provides the counterparty with a free option and freely speculate on the price of an asset being traded~\cite{han2019optionality}.

In this work, we describe an alternative approach for quick and efficient exchange of value between different blockchains.
Our approach orients around multi-signature transactions and requires witnesses to digitally sign transactions associated with a particular trade to enforce its atomicity.
A key advantage is that the majority of the computations required by our system do not involve expensive on-chain transactions but are instead off-chain, performed by users themselves.
By leveraging a lightweight accountability mechanism, we enable any participant to detect and prove the wrongdoings of a trader or witness, resulting in quick community exclusion.\todo{finish + add numbers}

\section{Problem Description}
This work focusses on the problem of two users wanting to securely exchange assets between different blockchain platforms.
Assume, for example, that Alice wants to buy Bitcoin for Ethereum while Bob wants to sell his Ethereum for Bitcoin.
Furthermore, assume that Alice and Bob have no prior trust relation (e.g., they do not know each others real-world identity).
Then, these two users reach an agreement to trade (e.g., containing the pricing information of the assets to be traded) and the process of exchanging assets, or \emph{settlement} can start.
A naive approach is to have both Alice and Bob issue a transaction where they transfer the agreed amounts of assets to the counterparty.
We refer to the transaction issued by Alice as $ tx_a $ and the transaction of Bob as $ tx_b $.
This unsupervised process, however, is risky: Alice or Bob can refrain from issuing the transaction, possibly resulting in a net loss for one of the parties.
The objective of an atomic swap is to prevent this exact situations: one transaction cannot complete without the other going through.

To address the lack of trust between traders, it is common to use a trusted intermediary, e.g., a cryptocurrency exchange, to process the trade.
The popularity of blockchain technology and different decentralized applications has resulted in the deployment of many cryptocurrency exchanges, some of them processing millions worth of transactions daily.
This approach requires Alice and Bob to trust that the intermediary correctly completes the trade and does not default or comprise the incoming assets.\todo{finish}
For an overview of state-of-the-art cross-chain trading mechanisms, we refer to reader to our prior work~\cite{de2021xchange}.\todo{list trust assumptions for each trading mechanism?}

As such, the overarching research question of this work is as follows: \emph{how can two users securely exchange assets between isolated blockchain platforms?}

\section{Efficient Asset Exchange using Witnesses}
We present an efficient approach to enable cross-chain asset exchange by leveraging multi-signature transactions and signature thresholds.
For presentation clarity, we first focus on permissioned networks where the identity of every user is well-established.

\section{Related Work}
TODO

\bibliographystyle{plain}
\bibliography{references}

\end{document}
